%%%%%%%%%%%%%%%%%%%%%%%%%%%%%%%%%%%%%%%%%%%%%%%%%%%%%%%%%%%%%%%%%%
%ANFANG VON FH VORLAGE
%%%%%%%%%%%%%%%%%%%%%%%%%%%%%%%%%%%%%%%%%%%%%%%%%%%%%%%%%%%%%%%%%%
%\chapter{FH-Vorlagen Archiv}

%\begin{figure}[!htbp]
%\centering
%\includegraphics[width=0.5\linewidth]{PICs/buchruecken}
%\caption{Beispiel für die Beschriftung eines Buchrückens.}\label{Abb1}
%\end{figure}

%\begin{figure}[!htbp]
%\centering
%\includegraphics[width=0.5\linewidth]{PICs/buchruecken}
%\caption{2. Beispiel für die Beschriftung eines Buchrückens.}\label{Abb2}
%\end{figure}

%\noindent Querverweise werden in \LaTeX{} automatisch erzeugt und verwaltet, damit sie leicht aktualisiert werden können. Hier wird zum Beispiel auf Abbildung %\ref{Abb1} verwiesen.
%Und hier ist ein Verweis auf Tabelle \ref{tab1}. Das gezeigte Tabellenformat ist nur ein Beispiel. Tabellen können individuell gestaltet werden.

%Etwas Text... Hier kommen noch einige Abkürzunge vor zum Beispiel \ac{ABC},\ac{WWW} und \ac{ROFL}.

%\begin{table}[!htbp]
%\centering
%\caption{Semesterplan der Lehrveranstaltung \glqq Angewandte Mathematik\grqq.}\label{tab1}
%\begin{tabular}{| p{0.3\linewidth} | p{0.3\linewidth} | p{0.3\linewidth} |}\hline
%Datum & Thema & Raum\\\hline
%20.08.2008 & Graphentheorie & HS 3.13\\
%01.10.2008 & Biomathematik & HS 1.05\\\hline
%\end{tabular}
%\end{table}
%\begin{table}[!htbp]
%\centering
%\caption{2. Semesterplan der Lehrveranstaltung \glqq Angewandte Mathematik\grqq.}\label{tab2}
%\begin{tabular}{| p{0.3\linewidth} | p{0.3\linewidth} | p{0.3\linewidth} |}\hline
%Datum & Thema & Raum\\\hline
%20.08.2008 & Graphentheorie & HS 3.13\\
%01.10.2008 & Biomathematik & HS 1.05\\\hline
%\end{tabular}
%\end{table}

%Hier wird auf die Formel \ref{Gl1} verwiesen.

%\begin{align}
%x = -\frac{p}{2}\pm\sqrt{\frac{p^2}{4}-q}\label{Gl1}
%\end{align}
%\begin{align}
%x = -\frac{p}{2}\pm\sqrt{\frac{p^2}{4}-q}\label{Gl2}
%\end{align}

%Literaturverweise sollten automatisch verwaltet werden, vor allem, wenn es viele Quellenverweise gibt. Beispiele sind  \cite{Ko05a}, \cite{Ko05b}, \cite{MiGo05}, \cite{TeGo14}, \cite{HuHa07}, \cite{HuZi10}, \cite{ZiKu07}, \cite{He07}, \cite{SIE11}, \cite{SIE14}, \cite{ISO98}, \cite{ATM11}, \cite{Hu11}, \cite{Po10}. Das verwendete Zitierformat (bzw.~das Format des Literaturverzeichnisses) ist entspechend der Vorgaben der Studiengänge zu wählen.
%Hier wird etwas Quellcode dargestellt:
%\begin{listing}[htbp]
%\begin{minted}[
%    frame=single,
%    framesep=2mm,
%    baselinestretch=1.2,
%    bgcolor=white,
%    fontsize=\footnotesize,
%%    linenos
%    ]{c}
%#include <iostream>
%
%void SayHello(void)
%{
%    // Kommentar
%    cout << "Hello World!" << endl;
%}
%
%int main(int argc, char **argv)
%{
%    SayHello();
%%%    return 0;
%%}
%\end{minted}
%\caption{Hello-World}
%\end{listing}
%
%
%\section{Algorithms}
%
%
%Use a defined environment for algorithms.
%
%Algorithm \ref{alg:euclid} is an example from the gallery (\url{https://www.overleaf.com/latex/examples/euclids-algorithm-an-example-of-how-to-write-algorithms-in-latex/mbysznrmktqf}) .
%\begin{algorithm}
%\caption{Euclid’s algorithm}\label{alg:euclid}
%\begin{algorithmic}[1]
%\Procedure{Euclid}{$a,b$}\Comment{The g.c.d. of a and b}
%\State $r\gets a\bmod b$
%\While{$r\not=0$}\Comment{We have the answer if r is 0}
%\State $a\gets b$
%\State $b\gets r$
%\State $r\gets a\bmod b$
%\EndWhile\label{euclidendwhile}
%\State \textbf{return} $b$\Comment{The gcd is b}
%\EndProcedure
%\end{algorithmic}
%%%%%%%%%%%%%%%%%%%%%%%%%%%%%%%%%%%%%%%%%%%%%%%%%%%%%%%%%%%%%%%%%%
%ENDE VON FH VORLAGE
%%%%%%%%%%%%%%%%%%%%%%%%%%%%%%%%%%%%%%%%%%%%%%%%%%%%%%%%%%%%%%%%%%